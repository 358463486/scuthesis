\chapter{模板使用说明}
\section{使用向导}
\subsection{文件结构}

\scuthesis{}为了不使主文件过大而使得写作过程中容易受到影响,
而建议采用表~\ref{tab:template-files} 建议的结构对论文文章进行分文件组织。

\begin{table}[htb]
  \centering
  \begin{minipage}[t]{.8\linewidth} 
  \caption{文件结构}
  \label{tab:template-files}
    \begin{tabularx}{\linewidth}{lX}
      \toprule[1.5pt]
      {\heiti 文件名} & {\heiti 描述} \\\midrule[1pt]
      scuthesis.cls & 模板类文件(\textbf{通常不需要修改})。\\
      scuthesis.cfg & 模板配置文件(\textbf{通常不需要修改})。\\
      scuthesis.bst & 参考文献BIB\TeX{}样式文件(\textbf{不建议修改})。\\
      scuthesis.sty & 常用的包和命令可以封装在这里。\\
			main.tex & 主文件,整合所有分散的文件于此并最后生成论文文档。\\
			chapters/basic.tex &  基本的个人信息。\\
			chapters/prologue.tex &  论文正文之前的包含\textbf{摘要}都放在这里。\\
			chapters/epilogue.tex &  论文正文之后的包含\textbf{致谢}都放在这里。\\
			chapters/chap01.tex &  论文正文第一章。\\
			chapters/appendix01.tex &  附录第一章。\\
      \bottomrule[1.5pt]
    \end{tabularx}
  \end{minipage}
\end{table}

在写更多的章节后,仅需要在\texttt{main.tex}文件中添加进去就可以了,如果对于 \LaTeX{} 不是很了解的可以直接参考模板文件的写作方式与风格。

\subsection{上手步骤}
首先应该先使用\XeLaTeX{}编译\texttt{main.tex}文件,确定基本环境搭建无误。
然后修改\texttt{chapters/basic.tex}文件,按照提示替换大括号内的内容,
填写个人基本信息,这一步完成后就可以看到\scuthesis{}已经定义为你的论文模板了。

其中注意,对于标题过长的,在定义过程中需要断行的地方加入一个$\sim$符号即可。

在\texttt{chapters/chap01.tex}中写你的正文第一章,
写第二章应当在\texttt{chapters} 文件夹下新建\texttt{chap02.tex}文件,
并在\texttt{main.tex}文件中的 \texttt{\textbackslash{}include\{chap01.tex\}}下将新的章节也\texttt{include}进来。
\scuthesis{}建议大家按照「章」分文件保存,
这样不会因所有内容堆积在一个文件内使得单个文件过于臃肿,
可以更专心与内容的输出,而按「节」存放过于琐碎,
此外如果想使用按「节」保存的同学需要注意 \texttt{\textbackslash{}inlcude}命令会自动换页,
所以按「节」保存的文件应当改用 \texttt{\textbackslash{}input}命令来引入。

完成正文后,可以在\texttt{chapters/prologue.tex}文件中完成中文和英文摘要,并填写关键词;
在\texttt{chapters/epilogue.tex} 文件中完成感谢内容;
在\texttt{chapters/appendix01.tex}中可以填写你的附录,
附录部分和正文一样按「章」分文件保存,
更多的章节需要自己创建文件并引入到\texttt{main.tex}文件中。

最后是参考文献的部分,\scuthesis{}采用BIB\TeX{}进行文献管理,
将文献按照BIB\TeX{}的规范按条目保存在\texttt{refs.bib}文件中,
在正文中需要插入参考文献的地方使用 \texttt{\textbackslash{}citep}来引入文献,文献序号以及最后的参考文献都会由\LaTeX{}自动生成。

\section{配置}
	\subsection{字体}

	\scuthesis{}中已经根据需求预设了本科生毕业论文需要的中文字体,如中易宋体等,以及西文字体Times New Roman等。

	如果现有的字体尚不足以满足你的需求,可以在 \texttt{ctex-fontset-scuthesis.def} 文件中添加,但是不建议修改目前预设字体,因为它们是必需的\citep{scdxbkbylwgshckwxzlyq_2006}。

\section{打印}
通常来说采用双面打印,其中封面单独胶封,而摘要单面打印,而在模板中页脚与页面都已经预设好了。此外,对于需要单面打印的可以添加\texttt{oneside}参数来使所有页面适合单面打印。
