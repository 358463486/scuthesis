\chapter{模板介绍}
\section{介绍}
	\scuthesis{}是为了帮助四川大学本科毕业生撰写本科毕业论文而编写的 \LaTeXe{}\citep{knuth_texbook_1986} 论文模板。
	
	该模板以C\TeX{}社区发布的ctexart模板为基础,以清华大学学位论文模板\textsc{ThuThesis}为参考而制作的,采用\XeLaTeX{}处理中文编码与字体问题,暂不支持 pdf\LaTeX{}。
	
	该模板根据2006年2月四川大学教务处发布的《四川大学本科毕业论文(设计)格式和参考文献著录要求》定制,除了不符合“用微软Word软件排式”的规定外,其它要求基本符合。
	
	该模板非四川大学官方模板,使用该模板可能引起的问题,\textbf{模板作者不承担任何责任},特此声明。建议在使用之前,先和自己的导师确认是否可以使用\LaTeX{}以及论文规范。 
	
\section{环境与基本设置}
\subsection{Mac操作系统}
建议直接安装Mac\TeX{}套件。
\subsection{Linux操作系统}
可以自行选择\LaTeX{}版本,但确认安装有C\TeX{}。
\subsection{Windows操作系统}
建议直接安装Mik\TeX{}套件。

\section{未来发展}
在2012届本科生中共有10名用户参与内测,并使用\scuthesis{}进行毕业论文撰写与提交,其中仅涵盖计算机学院与数学学院,
目前还不清楚其他学院的模板是否兼容,
如果不同,希望大家可以反馈不同学院的模板要求与样式,
\scuthesis{}会尽最大的可能兼容四川大学各个学院的本科生毕业论文模板。

\scuthesis{}目前已添加部分参数供用户根据自己的需求进行简单的调整,例如\texttt{openany}和\texttt{openright}来控制论文是否右开,
\texttt{oneside}和\texttt{twoside}参数以便单面打印或双面打印等,
在后续计划中可能会添加更多参数来满足大家不同的需求。

\textbf{在很远的将来},\scuthesis{}或许会添加对硕士论文以及博士论文的支持。
